\section{Parameters}
\label{sec:param}
This section describes various command line parameters that PEBBL
recognizes.  The listing is not exhaustive, but contains the
parameters you should typically find most useful.  A complete listing
is produced by specifying the \texttt{help} parameter to driver
programs constructed in the manner described in this guide.  You may
add your own application parameters as described in Section~\ref{sec:seradd}.


\subsection{Checkpointing}
\vspace{-3ex}
\pparamc{abortCheckpointCount}{int}{0}{Nonnegative} Primarily for
debugging purposes. Causes an abort after writing this many
checkpoints.  A zero value, which is the default, disables this feature.

\pparam{checkpointDir}{string}{Current directory, or from environment
variable} Directory to place checkpoint files.  The environment
variable \texttt{PEBBL\_CHECKPOINT\_DIR}, if defined, provides a
default value.  If this variable is undefined, the default is the process
current directory.

\pparamc{checkpointMinInterval}{double}{0}{Nonnegative}
Minimum minutes of CPU time per processor between writing
checkpoints.  

\pparamc{checkpointMinutes}{double}{0}{Nonnegative}
Desired minutes between starting to write successive checkpoints; the default
value of $0$ disables checkpointing.

\pparam{reconfigure}{bool}{\texttt{false}}
Resume from a previously written checkpoint, reading the checkpoint
files serially.  The configuration of worker and hub processors need
not be identical to the run that wrote the checkpoint.

\pparam{restart}{bool}{\texttt{false}} Restart from a previously saved
checkpoint, attempting to read the checkpoint files in parallel.  The
configuration of worker and hub processors must be identical to
the run that wrote the checkpoint.

\subsection{Debugging and performance tuning aids}
\label{sec:debugparams}
\vspace{-3ex}
\sparamc{debug}{int}{0}{Nonnegative}
Debugging diagnostic output level.  To add diagnostic output
controlled by this parameter to your application, you may use the
\texttt{DEBUGPR} and related macros.  

\sparam{debug-solver-params}{bool}{\texttt{false}}
If \texttt{true}, print the value of all parameters.

\pparamc{debugSeqDigits}{int}{0}{Lower bound: 0,  Upper bound: 10}
Number of sequence digits prepended to output lines.  This feature
allows you to run debug output through the unix \texttt{sort} utility
to obtain output grouped by processor.  Another way to separate
debugging output by processor is to use the \texttt{dumpSplit} 
utility described below in Section~\ref{sec:dumpsplit}.

\pparam{forceParallel}{bool}{\texttt{false}} Force the use of a parallel PEBBL
solver, even if there is only one processor.  

\pparam{hubDebug}{bool}{\texttt{false}}
Supplies a specific debug level for hub operations.

\pparam{loadLogSeconds}{double}{0} If positive, triggers output of
load logging information suitable for visualization with the 
\texttt{pebblLoadGraph} utility described in
Section~\ref{sec:pebblLoadGraph}.  A positive value indicates the
approximate number of seconds that should elapse between load information
samples on each processor.  For example, a value of 0.1 indicates that
each processor should record load information 10 times per second.

\sparam{printDepth}{bool}{\texttt{false}}
Include subproblem depth in debugging output.

\sparam{printIntMeasure}{bool}{\texttt{false}}
Include subproblem ``integrality measures'' in debugging output.

\pparam{workerDebug}{bool}{\texttt{false}}
Supplies a specific debug level for the worker thread and related operations.


\subsection{Enumeration}
\vspace{-3ex}
\sparamc{enumAbsTolerance}{double}{-1.0}{Lower bound: $-1$}
Absolute tolerance for enumeration.  Find solutions
that are within this additive distance of optimality.  The default
value means the feature should not be used.

\sparam{enumCutoff}{double}{(none)}
If set, specifies an absolute threshold on enumerated solutions.  Only
solutions with values strictly better than \texttt{enumCutoff} will be
retained. 

\sparamc{enumCount}{int}{0}{Nonnegative}
        If positive, indicates that only the best \texttt{enumCount}
        solutions (that meet other enumeration criteria, if any)
        should be retained. 

\sparamc{enumRelTolerance}{double}{-1.0}{Lower bound: $-1$}
        Relative tolerance for enumeration.  Find solutions
        that are within this multiplicative factor of being
        optimal.  For example, a value of $0.1$ requests solutions
        within 10\% of optimality.  The default value means the
        feature is disabled.


\subsection{General}
\vspace{-3ex}
\sparam{help}{bool}{\texttt{false}}
If \texttt{true}, print usage information and parameter definitions, and then exit.

\sparamc{randomSeed}{int}{1}{nonnegative}
Global seed for random number generation.

\sparam{version}{bool}{\texttt{false}}
If \texttt{true}, print version information and exit. Should be used  on
the command line only, and as the first parameter specified.


\subsection{Incumbent}
\vspace{-3ex}
\sparam{startIncumbent}{double}{(none)}
Value of some known feasible solution.

\subsection{Output}
\label{sec:outputparams}
\vspace{-3ex}
\sparamc{earlyOutputMinutes}{double}{0}{Nonnegative} 
If this many minutes have elapsed since its creation,
output the current incumbent to a file in case of
a crash or timeout.  The default value disables this
feature.

\sparam{output}{CharString}{See Section~\ref{sec:serialdriver}}
File in which to write the final solution(s).  If omitted, output is
written in the current directory, using a filename constructed as
described in Section~\ref{sec:serialdriver}.

\sparam{printFullSolution}{bool}{\texttt{false}}
Print full solution(s) to standard output as well as writing it to a file.

\pparam{printSolutionSynch}{bool}{\texttt{true}} Indicates that only
MPI's designated I/O processor (typically processor 0) is allowed to
write the solution.  If \texttt{false}, some communication overhead
may be saved in the solution output process.

\sparamc{statusPrintCount}{int}{100,000}{Nonnegative}
The maximum number of subproblems bounded between status printouts.
Status printouts are triggered by thresholds on wall clock time and the
total number of subproblems bounded since the last status printout.
Since the default value is large, the default behavior will typically
be to trigger status printouts based on wall clock time.

\sparamc{statusPrintSeconds}{double}{10.0}{Nonnegative}
The maximum number of seconds elapsing between status printouts.

\sparam{suppressWarnings}{bool}{\texttt{false}}
Suppress warning messages.

\pparam{trackIncumbent}{bool}{\texttt{false}}
Print a message whenever there is a new incumbent.

\subsection{Parallel work distribution}
\vspace{-3ex}
\pparamc{clusterSize}{int}{64}{Lower bound: 1}
Maximum number of processors controlled by a single hub (including the
hub itself).  Unless \texttt{numClusters} is set, this will be the
size of all but the last cluster.

\pparamc{hubsDontWorkSize}{int}{10}{Lower bound: 2}
Size of cluster at or above which hubs do not also function as
workers.

\pparamc{hubLoadFac}{double}{0.10}{Lower bound: 0 , Upper bound: 1}
The target fraction of subproblems to be controlled by hub
processors at any given time.  See Section~\ref{sec:withincluster} for
a detailed discussion of the funcion of this parameter.

\sparamc{loadMeasureDegree}{int}{1}{Must be 0, 1, 2, or 3} 
Used to
measure the ``weight'' of a subproblem used to calculate worker and
cluster workloads.  Specifically, if a subproblem has bound $b$ and
the current incumbent value is $z$, its weight is ${\left|z -
b\right|}^{\text{\texttt{loadMeasureDegree}}}$.  Note that subproblem
weight is used primarily by the load balancing algorithms in the
parallel layer, but \texttt{LoadMeasureDegree} also exists in the serial
layer for technical reasons.  
Its value should not materially affect the operation of the serial
layer.  In the parallel layer, larger values put more load balancing
stress on the quality of subproblems, and lower values more stress on
quantity of subproblems; a value of zero sets a processor's workload
equal to the number of subproblems it controls.  The parameter
\texttt{qualityBalance} specifies additional attention to subproblem
quality in load balancing, beyond that specified in
\texttt{loadMeasureDegree}. 
 
\pparamc{numClusters}{int}{1}{Lower bound: 1}
Forces a minimum number of processor clusters, even if all are smaller than
\texttt{clusterSize}. 

\pparam{qualityBalance}{bool}{\texttt{true}}
If \texttt{true}, hubs shift work between workers based on each
worker's best subproblem bound,
as well as the total workload as measured by~\eqref{loadcalc} in
Section~\ref{sec:betweenclusters}.  Note that this workload metric
may already take some measure of quality into account if
\texttt{loadMeasureDegree} is positive.

\pparamc{minScatterProb}{double}{0.05}{Lower bound: 0 , Upper bound: 1}
\groupparams
\pparamc{targetScatterProb}{double}{0.25}{Lower bound: 0,  Upper bound: 1}
\groupparams
\pparamc{maxScatterProb}{double}{0.90}{Lower bound: 0,  Upper bound:
  1}
%%%% JE this last one went away
%% \groupparams
%% \pparamc{targetWorkerKeepFrac}{double}{0.70}{Lower bound: 0,  Upper bound: 1}
These parameters control
how frequently subproblems are released from workers.  Generally, you
should observe the ordering
$$
\text{\texttt{minScatterProb}} \leq
\text{\texttt{targetScatterProb}} \leq
\text{\texttt{maxScatterProb}}.
$$ 
The release decision scheme uses the notion of a worker having its
``fair share'' of work, as described in
Section~\ref{sec:withincluster}.  If a
worker has exactly its fair share of work, then it releases subproblems with
probability \texttt{targetScatterProb}.  If it has more than its fair
share, it uses a higher probability, linearly increasing up to
\texttt{maxScatterProb} if it has 100\% of the work in the system.
Similarly, if it has less than its fair share, it uses a lower
probability, linearly decreasing down to \texttt{minScatterProb} if
appears to have no work.

\pparamc{minNonLocalScatterProb}{double}{0.0}{Lower bound: 0,  Upper bound: 1}
\groupparams
\pparamc{targetNonLocalScatterProb}{double}{0.33}
        {Lower bound: 0, Upper bound: 1}
\groupparams
\pparamc{maxNonLocalScatterProb}{double}{default: 0.9}
        {Lower bound: 0, Upper bound: 1}
When there is more than one cluster, these parameters control the
decision, once a worker has decided to release a subproblem, of
whether it should be released to the worker controlling hub, or to a
randomly chosen hub (which could also be the worker's own hub).  This
decision is based on whether the worker's cluster has its ``fair
share'' of the total workload: a fraction $w(c)/W$ of the total work,
where $w(c)$ is the number of workers in the worker's cluster, and $W$
is the total number of workers.  When the worker's cluster has its
fair share, random scattering is performed with probability
\texttt{targetNonLocalScatterProb}.  If the cluster has more than its
fair share, a larger probability is used, increasing linearly to
\texttt{maxNonLocalScatterProb} if the cluster has all the work in the
system.  If the cluster has less work, a smaller probability is used,
decreasing linearly to \texttt{minNonLocalScatterProb} if the cluster
has no work.


\subsection{Parallel thread control}
\label{sec:pthread}
\vspace{-3ex}
\pparamc{incThreadBiasFactor}{double}{100.0}{Nonnegative}
\groupparams
\pparamc{incThreadBiasPower}{double}{1.0}{Nonnegative}
\groupparams
\pparamc{incThreadMaxBias}{double}{20.0}{Nonnegative}
\groupparams
\pparamc{incThreadMinBias}{double}{1.0}{Nonnegative}
These parameters are used in computing the bias (priority) of the
incumbent heuristic thread, if it is present.  If we represent the
above parameters by $\phi$, $\pi$, $\overline{b}$, and
$\underline{b}$, respectively, the formula for the incumbent thread
bias $b$ is
$$
b = \max\left\{\underline{b},
    \min\left\{\overline{b}, \phi r^{\pi} \right\} \right\},
$$ 
where $r$ is the current relative gap, that is, the relative
difference between the incumbent value and the best known bound in the
pool of active subproblems.

\pparamc{timeSlice}{double}{0.01}{Lower bound: $10^{-7}$}
Target thread timeslice in seconds.  This is the typical
run time or ``granularity'' that the scheduler tries to acheive for
each invocation of a compute thread.

\pparam{useIncumbentThread}{bool}{\texttt{true}}
Controls whether each worker dedicates a thread to incumbent search.
If \texttt{false}, the only source of new incumbents will be terminal
subproblems and  \texttt{quickIncumbentHeuristic()}.  This
parameter is ignored if \texttt{haveParallelIncumbentHeuristic()}
returns \texttt{false}.

\pparamc{workerThreadBias}{double}{100.0}{Nonnegative}
Scheduling priority for main worker thread.


\subsection{Ramp-up (standard implementation)}
The following parameters control the default implementation of the
ramp-up crossover mechanism, that is, the transition from ramp-up to
``regular'' parallel execution.  Your application may override
\texttt{parallelBranching}'s default implementations of
\texttt{continueRampUp()} or \texttt{forceContinueRampUp()} to ignore
these parameters or interpret them differently.

\pparamc{minRampUpSubprobsCreated}{int}{0}{Nonnegative}
Force this many subproblem creations before ramp up ends.

\pparamc{rampUpPoolLimit}{int}{0}{Nonnegative}
Total subproblem pool size beyond which the ramp-up phase may end.

\pparamc{rampUpPoolLimitFac}{double}{1}{Nonnegative}
Desired average number of subproblems per worker processor
immediately after ramp-up.


\subsection{Search order and protocol}
\label{sec:searchparams}
\vspace{-3ex}
\sparam{breadthFirst}{bool}{\texttt{false}}
In serial, use breadth-first search; in parallel, use an approximation
of breadth-first search based on treating all subproblem pools as FIFO
queues.  Ignored if \texttt{depthFirst} is also specified.

\sparam{depthFirst}{bool}{\texttt{false}} 
In serial, use depth-first search; in
parallel, use an approximation based on treating all
subproblem pools as stacks.  Overrides \texttt{breadthFirst} if both
are specified.

\vspace{2ex}

Note that if neither \texttt{breadthFirst} nor \texttt{depthFirst} is
specified, then PEBBL uses best-first search.  That is, it tries to
select the subproblem with the lowest possible bound for minimization
problems, and the highest possible bound for maximization problems.
In parallel, conformance to the best-first search order is only approximate.

\sparam{initialDive}{bool}{\texttt{false}} 
\groupparams
\sparam{integralityDive}{bool}{\texttt{true}} 
These options are useful for
applications that do not have a good incumbent heuristic.  Setting
\texttt{initialDive} is incompatible with the \texttt{breadthFirst}
and \texttt{depthFirst} options, and \texttt{integralityDive} is only
meaningful if \texttt{initialDive} is \texttt{true}.
\texttt{InitialDive} specifies that the best-first search order should
only be followed after the first incumbent is found.  Beforehand,
PEBBL should ``dive'' in the tree to try to identify a feasible
solution.  If \texttt{integralityDive} is \texttt{true}, then
``diving'' means giving priority to processing subproblems with the
\emph{lowest} value of the data member
\texttt{branchSub::integralityMeasure} (a value of zero 
is interpreted as meaning a subproblem is
``integer feasible'').  If \texttt{integralityDive} is \texttt{false},
it means selecting the subproblem with the highest possible depth.
Both these techniques tend to produce initial search trees similar to
classical depth-first search.  Once an incumbent is found, PEBBL
reverts to best-first search.

\sparam{eagerBounding}{bool}{\texttt{false}} 
Specifies the search protocol
implemented by the ``eager'' handler, as described in
Section~\ref{sec:framework}.  This handler tries to bound subproblems
as soon as they are created and keep all subproblems in the pool in
either the \texttt{bounded} or possibly \texttt{beingBounded} or
\texttt{beingSeparated} states.  This handler is recommended for
applications in which subproblem bounds are typically computed very
quickly.  This parameter is ignored if \texttt{lazyBounding} is also
specified.

\sparam{lazyBounding}{bool}{\texttt{false}} 
Specifies the search protocol
implemented by the ``lazy'' handler, as described in
Section~\ref{sec:framework}.  This handler attempts to delay bounding
subproblems as long as possible, and fill all subproblem pools with
\texttt{boundable} and possibly \texttt{beingBounded} or
\texttt{beingSeparated} problems.  Once a problem is
\texttt{separated}, its children are created as soon
as possible.  This option takes precedence over
\texttt{eagerBounding}. 

\vspace{2ex}

If both \texttt{eagerBounding} and \texttt{lazyBounding} are
\texttt{\texttt{false}}, which is the default, PEBBL uses the ``hybrid''
handler described in Section~\ref{sec:framework}.  This handler simply
chooses problems from the subproblem pool, attempts to advance them
one state in Figure~\ref{fig:states}, and replaces them.  With this handler,
the subproblem pools may contain a mix of all the possible states
except \texttt{dead}.

\subsection{Termination}
\vspace{-3ex} 
\sparamc{absTolerance}{double}{0}{Nonnegative} 
When enumeration is not active, the absolute
tolerance for optimal objective value.
Subproblems are fathomed when
their bounds are within \texttt{absTolerance} of the incumbent value.
The final solution reported should be within this distance of the true
optimum.  This parameter is used to set the \texttt{branching} class
member \texttt{absTol}.  However, applications may adjust the value
of \texttt{absTol}.  For example, in a linear pure integer program
with all objective function coefficients integer, you may want to set
\texttt{absTol} to at least $1$.  In this situation, you could
alternatively design your \texttt{branching}-derived class to round up
the value of \texttt{branchSub::bound} to the next integer.

\sparamc{relTolerance}{double}{$10^{-7}$}{Nonnegative}
When enumeration is not in use, the 
relative tolerance for optimal objective value.  If the incumbent
value is $z$ and the subproblem bound is $b$, a subproblem can be
fathomed if ${\left|(z - b)/z\right|} \leq
\text{\texttt{relTolerance}}$.  This parameter essentially controls the
number of digits of precision of the final solution.  A
value of zero is possible, but not recommended unless your bound
calculations use only integer arithmetic, and there is no possibility
of round-off error.  This parameter is used to set the data member
\texttt{relTol} in the \texttt{branching} class.

\sparamc{integerTolerance}{double}{$10^{-5}$}{Lower bound: 0,  Upper bound: 1}
Tolerance for determining whether numbers are integers.  Note that
this parameter does not affect PEBBL itself, but only applications that
use the methods \texttt{isInteger(double)} or
\texttt{isZero(double)} provided for convenience in class
\texttt{pebblBase}, from which \texttt{branching} and
\texttt{branchSub} are both derived.

\sparamc{maxCPUMinutes}{double}{0}{Nonnegative}
\groupparams
\sparamc{maxSPBounds}{double}{0}{Nonnegative}
\groupparams
\sparamc{maxWallMinutes}{double}{0}{Nonnegative}
These parameters control PEBBL's built-in abort function.  A PEBBL run
will abort if the CPU time (per processor) exceeds
\texttt{maxCPUMinutes}, the total number of subproblems bounded
exceeds \texttt{maxSPBounds}, or the total wall clock time spent in
the search exceeds \texttt{maxWallMinutes}.  In each case, a zero
value, which is the default, means there is no limit.

\sparam{printAbortMessage}{bool}{\texttt{true}}
Instructs PEBBL to print an explanatory message when
aborting due to the \texttt{maxCPUMinutes}, \texttt{maxSPBounds}, or
\texttt{maxWallMinutes} limits.

\sparam{rampUpOnly}{bool}{\texttt{false}}
Forces PEBBL runs to terminate immediately after ramp-up.  This
parameter is provided primarily for debugging or evaluating the
performance of your application's ramp-up phase.

\sparam{useAbort}{bool}{\texttt{false}}
If \texttt{true}, force an abort when an error occurs.


